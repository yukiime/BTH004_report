\documentclass[UTF8]{ctexart}

\renewcommand{\contentsname}{Content}

\usepackage{geometry}
\usepackage{graphicx}
\usepackage{listings}
\usepackage{algorithm}
\usepackage{color}

\usepackage{algorithmic}
\usepackage[colorlinks,linkcolor=blue]{hyperref}

\definecolor{dkgreen}{rgb}{0,0.6,0}
\definecolor{gray}{rgb}{0.5,0.5,0.5}
\definecolor{mauve}{rgb}{0.58,0,0.82}
\lstset{ %
    aboveskip=3mm,
    belowskip=3mm,
    showstringspaces=false,
    columns=flexible,
    basicstyle={\small\ttfamily},
    numbers=left,
    numberstyle=\tiny\color{gray},
    keywordstyle=\color{blue},
    commentstyle=\color{dkgreen},
    stringstyle=\color{mauve},
    breaklines=true,
    breakatwhitespace=true,
    tabsize=3
}
\geometry{left=3.18cm,right=3.18cm,top=2.54cm,bottom=2.54cm}
\pagestyle{plain}   
% \usepackage{booktabs}
% \usepackage{subfigure}
\usepackage{setspace}
\date{}
\begin{document}

\begin{center}
    \quad \\
\end{center}
\vskip 3.5cm

\begin{center}
    \quad \\
    \quad \\
    % \huge  assignment1
    \fontsize{36pt}{27pt}\selectfont \textbf{assignment1}
    \quad \\
\end{center}
\vskip 1.0cm

\begin{center}
    % \includegraphics[scale=0.6]{../img/logo.png}
    \fontsize{27pt}{27pt}\selectfont \textbf{Individual report for bth004}
\end{center}
\vskip 4.5cm

\begin{quotation}
    % \fontsize{27}{15}
    \fontsize{27}{15}\selectfont
    \doublespacing
    \par\setlength\parindent{6.5em}
    \quad

    Name: {\qquad\qquad\qquad YuKi  \qquad\qquad\qquad\qquad }

    Student ID: {\qquad\qquad xxxxxxxxxxxxx \qquad\qquad\qquad}

    Date: : {\qquad\qquad\qquad  December 4, 2023 \qquad\qquad\qquad}

    \vskip 1cm
    \centering
\end{quotation}

\newpage
\tableofcontents

\newpage
\section{Greedy Algorithm}
\subsection{Implementation}
In this question, an object-oriented approach is adopted to construct 
items and knapsack into $Item$ class and $Knapsack$ class respectively.

Then proceed as follows:
\begin{enumerate}
    \item Get the item list and knapsack list, 
    encapsulate them into instances, and store them in the $Item$ list and $Knapsack$ list respectively.
    \item For $Item$ list, the unit weight value of each item is calculated and stored.
    \item Sort the $Item$ list according to unit weight value from largest to smallest.
    \item Sort the $Knapsack$ list according to the remaining capacity of the knapsack from small to large.
    \item Each step, put the item with the largest unit weight value into the knapsack with the smallest remaining capacity. 
    After placing it, repeat 4 (reorder the $Knapsack$ list).
    \item Repeat step 5 until all knapsacks are full and no more items can be placed in knapsacks.
\end{enumerate}
% \newpage

\subsection{Pseudo Code}
\begin{algorithm}
    \caption{Greedy Algorithm}
    \label{alg1}
    \begin{algorithmic}[1]
        \REQUIRE Value of $n$ items, Weight of $n$ items, Capacity of $m$ knapsacks
        \ENSURE Maximum value sum $Z$, Item placement status
        \STATE \textbf{class} Item:
        \STATE \quad properties: sequenceNo, value, weight, benefit
        \STATE \textbf{class} Knapsack:
        \STATE \quad properties: sequenceNo, capacity, residualCapacity, items
        \STATE 
        \STATE $items\_list$ = [Item($i$+1,value,weight,$\frac{value}{weight}$) for $i$ in range($n$)]
        \STATE $knapsacks\_list$ = [Knapsack($i$+1,capacity) for $i$ in range($m$)]
        \STATE 
        % \STATE \textbf{function} \textit{sort\_by\_benefit\_desc}(items\_list)
        % \STATE \quad return $sort$($items\_list$, key=lambda item: item.benefit, reverse=True)
        % \STATE \textbf{function} \textit{sort\_by\_residual\_capacity\_asc}(knapsacks\_list)
        % \STATE \quad return $sort$($knapsacks\_list$, key=lambda knapsack: knapsack.residualCapacity)
        % \STATE
        \STATE \textbf{call} \textit{sort\_by\_benefit\_desc}($items\_list$)
        \FOR {$item$ in $items\_list$}
        \STATE \textbf{call} \textit{sort\_by\_residual\_capacity\_asc}($knapsacks\_list$)
        \FOR {$knapsack$ in $knapsacks\_list$}
        \IF {$item$ not in $knapsack$ $\&\&$ $knapsack.residualCapacity$ $\geq$ $item.weight$}
        \STATE put $item$ in $knapsack$
        \ENDIF
        \ENDFOR
        \ENDFOR
        \STATE \textbf{Output} Maximum value sum $Z$, Item placement status $knapsacks\_list$
        % \STATE \textbf{Output} Item placement status $knapsacks\_list$
    \end{algorithmic}
\end{algorithm}
% \newpage

\subsection{Verify Correctness}
\label{GreedyCorrectness}
Verify correctness using test case
% data1
\subsubsection{Test Data 1}
\noindent \textbf{Input: }\\
\includegraphics{img/testData1_Input.png}\\
\textbf{Output: }\\
\includegraphics{img/testData1_Output.png}
% data2
\subsubsection{Test Data 2}
\noindent \textbf{Input: }\\
\includegraphics{img/testData2_Input.png}\\
\textbf{Output: }\\
\includegraphics{img/testData2_Output.png}
% data3
\subsubsection{Test Data 3}
\noindent \textbf{Input: }\\
\includegraphics{img/testData3_Input.png}\\
\textbf{Output: }\\
\includegraphics{img/testData3_Output.png}

\subsection{Python Code}
\subsubsection{knapsack\_greedy.py}
Code for implementing a greedy algorithm to solve the knapsack problem
\lstinputlisting[language=Python]{code/knapsack_greedy.py}

\subsubsection{TestDataGenerator.py}
Code used to generate test data
\lstinputlisting[language=Python]{code/TestDataGenerator_greedy_forTex.py}
\newpage

\section{neighbourhood Search Algorithm}
\label{neighbourhood}
\subsection{Implementation}
In this question, we first obtain an initial solution, and express and store it in the form of a matrix. 
Its row(ith) is the knapsack serial number(i+1), and its column(jth) is the item serial number(j+1).

Then we stipulate that solution[i][j]=1 means that item j is placed in knapsack i. 
Otherwise, solution[i][j]=0 means that item j is not placed in knapsack i.

\label{DefineNeighbourhood}
Then For a solution $\hat x$, define a neighbourhood $N (\hat x) $: 
\begin{enumerate}
    \item Put an item j that is not in knapsack into knapsack i
    \item If the capacity of knapsack i is not enough, 
    move item jj (weight greater than item j) form knapsack i to the other knapsack ii.
    \item If the capacity of knapsack ii is not enough, 
    move the item jjj form knapsack ii out so that the item jj can be moved into knapsack ii.
\end{enumerate}

Then we use the initial solution to enter loop:
\begin{enumerate}
    \item Perform small modifications on the solution to get the neighbour in the neighbourhood.
    \item To evaluate all neighbours and select the best one.
    \item If the best neighbour is better than the current solution, 
    it will be used as the solution for the next iteration. 
    If it is not better than the current solution, break out of the loop.
\end{enumerate}

The solution after the loop ends is the optimal solution obtained by the neighbourhood search algorithm.

\subsection{{Pseudo Code}}
\begin{algorithm}
    \caption{neighbourhood search algorithm}
    \label{alg2}
    \begin{algorithmic}[1]
        \REQUIRE Value of $n$ items, Weight of $n$ items, Capacity of $m$ knapsacks, initial $solution$
        \ENSURE Maximum value sum $Z$, optimal $solution$
        \WHILE {No optimal $solution$ found}
        \STATE $Z \gets $ sum of items' values of $solution$
        \STATE $neighbourhood \gets $ \textit{getNeighbourhood}($solution$)
        % \STATE $Z\_bestNeighbour, bestNeighbour \gets $ \textit{evaluate}($neighbourhood$)
        
        \STATE $Z\_bestNeighbour, bestNeighbour \gets $ \textit{evaluate}($neighbourhood$)
        % \STATE $Z\_bestNeighbour \gets $ sum of items' values of $bestNeighbour$
        \IF {$Z\_bestNeighbour$ $\geq$ $Z$}
        \STATE $solution \gets bestNeighbour$
        \ELSE
        \STATE \textbf{break}
        \ENDIF
        \ENDWHILE
        \STATE \textbf{Output} Maximum value sum $Z$, optimal $solution$
    \end{algorithmic}
\end{algorithm}

\subsection{Verify Correctness}
Verify correctness using test case.
The three sets of test cases are the same as in the previous 
\hyperref[GreedyCorrectness]{\textit{1.3 Verify Correctness}}

% data1
\subsubsection{Test Data 1}
\noindent \textbf{Input: }\\
\includegraphics{img/testData1_Input2.png}\\
\textbf{Output: }\\
\includegraphics{img/testData1_Output2.png}
% data2
\subsubsection{Test Data 2}
\noindent \textbf{Input: }\\
\includegraphics{img/testData2_Input2.png}\\
\textbf{Output: }\\
\includegraphics{img/testData2_Output2.png}
%data3
\subsubsection{Test Data 3}
\noindent \textbf{Input: }\\
\includegraphics{img/testData3_Input2.png}\\
\textbf{Output: }\\
\includegraphics{img/testData3_Output2.png}

\subsection{Python Code}
\subsubsection{knapsack\_neigbour.py}
Code for implementing a neighbourhood search algorithm to solve the knapsack problem
\lstinputlisting[language=Python]{code/knapsack_neigbour.py}

\subsubsection{TestDataGenerator.py}
\label{TestDataGenerator_nt_forTex}
Code used to generate test data
\lstinputlisting[language=Python]{code/TestDataGenerator_nt_forTex.py}
\newpage

\section{Tabu Search Algorithm}
\subsection{Implementation}

The tabu-search algorithm, 
which are based on 
\hyperref[neighbourhood]{\textit{neighbourhood search algorithm}}, 
creates a k-length tabu list 
that holds $k$ most recent solutions which are tabu for subsequent search.  
And if the tabu list is filled, then tabu list will be remove the first solution.

So in this question, its neighbourhood definition is exactly the same as
\hyperref[DefineNeighbourhood]{\textit{neighbourhood search algorithm}}. 
While, we take the maximum size of the neighbourhood as the $k$.

For never getting stuck in a locally optimal solution, 
we need to record the globally optimal solution. 
Then we use the initial solution to enter loop:
\begin{enumerate}
    \item Update tabu list.
    \item Perform small modifications on the solution to get the neighbour in the neighbourhood.
    \item To evaluate all neighbours which are not in tabu list and select the best one.
    \item If the best neighbour is better than the globally optimal solution and, 
    it will be saved as the globally optimal solution.
    \item Use the best neighbour as the solution for the next iteration. 
    \item If the best neighbour found in step 2 do not exist, 
    that is, all neighbours are all in the tabu list, 
    break out of the loop.
\end{enumerate}
The globally optimal solution after the loop ends 
is the optimal solution obtained by the tabu search algorithm.
\newpage

\subsection{Pseudo Code}
\begin{algorithm}
    \caption{Tabu search algorithm}
    \begin{algorithmic}[1]
        \REQUIRE Value of $n$ items, Weight of $n$ items, Capacity of $m$ knapsacks, initial $solution$
        \ENSURE Maximum value sum $Z$, globally optimal $solution\_globally\_optimal$
        \STATE \textbf{function} \textit{updateTabuList}($tabuList, solution$)      
        \STATE \quad \textbf{if} {$tabuList$ not filled} \textbf{then}
        \STATE \quad\quad remove $tabuList[0]$
        \STATE \quad \textbf{end if}
        \STATE \quad $tabuList.append(solution)$
        % \STATE         
        % \STATE \textbf{function} \textit{filter\_by\_tabuList}($neighbourhood$)   
        % \STATE \quad  $neighbourhood\_notIn\_tabuList \gets neighbourhood$
        % \STATE \quad \textbf{for} {$neighbour$} \textbf{in} $neighbourhood$
        % \STATE \quad \quad \textbf{if} {$neighbour$ \textbf{in} $tabuList$} \textbf{then}
        % \STATE \quad\quad \quad $neighbourhood\_notIn\_tabuList.remove(neighbour)$
        % \STATE \quad \quad \textbf{end if}
        % \STATE \quad \textbf{end for}
        % \STATE \quad \textbf{return} $neighbourhood\_notIn\_tabuList$
        % \STATE
        \STATE \textbf{function} \textit{evaluate}($neighbourhood$)   
        \STATE \quad \textbf{for} {$neighbour$} \textbf{in} $neighbourhood$
        \STATE \quad \quad \textbf{if} {$neighbour$ \textbf{not in} $tabuList$} \textbf{then}
        \STATE \quad \quad \quad {$Z\_Neighbour \gets $ sum of items' values of $neighbour$}
        \STATE \quad \quad \quad \textbf{if} {$Z\_Neighbour$ $\geq$ $Z\_bestNeighbour$} \textbf{then}
        \STATE \quad \quad \quad \quad $Z\_bestNeighbour \gets $ $Z\_Neighbour$
        \STATE \quad \quad \quad \quad $bestNeighbour \gets $ $neighbour$
        \STATE \quad \quad \quad \textbf{end if}
        \STATE \quad \quad \textbf{end if}
        \STATE \quad \textbf{end for}
        \STATE \quad \textbf{return} $Z\_bestNeighbour, bestNeighbour$
        \STATE 
        \STATE $Z \gets $ sum of items' values of $solution$
        \STATE $k \gets n^2m$
        \STATE $tabuList \gets [\quad]$
        % \STATE 
        \WHILE {$True$}
        \STATE \textbf{call} $updateTabuList(tabuList, solution)$
        \STATE $neighbourhood \gets $ \textit{getNeighbourhood}($solution$)
        % \STATE $neighbourhood\_notIn\_tabuList \gets $ \textit{filter\_by\_tabuList}($neighbourhood$)
        
        \STATE $Z\_bestNeighbour, bestNeighbour \gets $ \textit{evaluate}($neighbourhood$)
        
        \IF {$Z\_bestNeighbour, bestNeighbour$ exist}
        \IF {$Z\_bestNeighbour$ $\geq$ $Z$}
        \STATE $Z \gets Z\_bestNeighbour$ 
        \STATE $solution\_globally\_optimal  \gets bestNeighbour$
        \ENDIF

        \STATE $solution \gets bestNeighbour$
        \ELSE
        \STATE \textbf{break}
        \ENDIF
        \ENDWHILE
        \STATE
        \STATE \textbf{Output} Maximum value sum $Z$, globally optimal $solution\_globally\_optimal$
    \end{algorithmic}
\end{algorithm}
\newpage

\subsection{Verify Correctness}
Verify correctness using test case.
The three sets of test cases are the same as in the previous 
\hyperref[GreedyCorrectness]{\textit{1.3 Verify Correctness}}

% data1
\subsubsection{Test Data 1}
\noindent \textbf{Input: }\\
\includegraphics{img/testData1_Input3.png}\\
\textbf{Output: }\\
\includegraphics{img/testData1_Output3.png}
% data2
\subsubsection{Test Data 2}
\noindent \textbf{Input: }\\
\includegraphics{img/testData2_Input3.png}\\
\textbf{Output: }\\
\includegraphics{img/testData2_Output3.png}
%data3
\subsubsection{Test Data 3}
\noindent \textbf{Input: }\\
\includegraphics{img/testData3_Input3.png}\\
\textbf{Output: }\\
\includegraphics{img/testData3_Output3.png}

\subsection{Python Code}
\subsubsection{knapsack\_tabu.py}
Code for implementing a tabu search algorithm to solve the knapsack problem
\lstinputlisting[language=Python]{code/knapsack_tabu.py}

\subsubsection{TestDataGenerator.py}
Code used to generate test data,
it is exactly the same as the code in
\hyperref[TestDataGenerator_nt_forTex]{\textit{2.4.2 TestDataGenerator}}, 
so it is omitted
% \lstinputlisting[language=Python]{code/TestDataGenerator_nt_forTex.py}

\end{document}